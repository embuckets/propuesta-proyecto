%Este comando hace que la página que contiene la introducción sea la número uno

%\tableofcontents

\setcounter{page}{1}
\section{Introducción}%Introducción
Un agente de seguros es la persona física o moral autorizada por la Comisión Nacional de Seguros y Fianzas para realizar actividades de intermediación en la contratación de seguros o de fianzas. Las actividades de intermediación que pueden realizar los agentes consiste en el intercambio de propuestas, comercialización y asesoramiento para la contratación de seguros o fianzas, su conservación o modificación, renovación o cancelación \cite{www:reg-agentes}.

Existe una gran variedad de ramos de pólizas (vida, gastos médicos, autos, daños, entre otros) cada uno con condiciones y coberturas distintas. El asegurado puede elegir el tipo de pago fraccionado (mensual, trimestres, semestral, anual) y en algunas pólizas la vigencia del plan.

El agente debe asesorar al asegurado en el uso de su póliza, además de recordarle el pago de su recibo y fecha de renovación.

Con los años los agentes de seguros pueden llegar a tener cientos de clientes y llevar el control de las pólizas, sin una aplicación, se vuelve complicado debido a que es una actividad manual.

En este proyecto se propone automatizar dicha tarea mediante una aplicaci\'on para el Control de Cartera para Agentes de Seguros que tiene por objetivo facilitar \'esta tarea.

\section{Justificación}

No existe software libre dedicado al control de cartera que los agentes de seguros puedan utilizar para este aspecto del trabajo, por lo tanto, tienen que recurrir a sus propios métodos. La mayoría utiliza hojas de cálculo o un sistema de archivos, lo que hace el mantenimiento de cartera una tarea muy manual, misma que consume tiempo del agente.
Una aplicación para el Control de Cartera, como software libre, facilita la gestión de clientes, pólizas, cobranzas y renovaciones; lo cual tendría un impacto en la productividad del agente.

\section{Objetivos}
\addcontentsline{toc}{subsection}{Objetivo General}
\subsection*{Objetivo General}
Desarrollar un Sistema de Control de Cartera para agentes de seguros que disminuirá la complejidad de la gestión de cartera.
\addcontentsline{toc}{subsection}{Objetivos Específicos}
\subsection*{Objetivos Específicos}
\begin{enumerate}
	\item Gestionar clientes, pólizas, cobranza y renovaciones.
	\item Guardar documentos de los clientes y formatos de las aseguradoras.
	\item Reportar proyecciones de comisiones.
\end{enumerate}

\section{Trabajos Relacionados}
\addcontentsline{toc}{subsection}{Software Comercial}
\subsection*{Software Comercial}
\addcontentsline{toc}{subsubsection}{SICAS}
\subsubsection*{SICAS \cite{www:sicas}} 

SICAS Online es un sistema Web para el Control y Administración para Cartera de Agentes, Corredores, Promotores de Seguros\footnote{Director de una agencia. Una agencia o promotor\'ia es una oficina que agrupa a varios agentes.} y afines. Ayuda en todo el proceso de administración aumentando la productividad reduciendo la captura y reutilizando información.

\addcontentsline{toc}{subsubsection}{Insly}
\subsubsection*{Insly \cite{www:insly}}

Software para agencias de seguros basado en la nube. Permite administrar el flujo de ventas gestionando clientes, pólizas y productos de seguros. 

\addcontentsline{toc}{subsubsection}{Asesorestic}
\subsubsection*{Asesorestic – Software para Administración de pólizas de seguros \cite{www:asesorestic}}

El sistema contribuye a tener el control sobre el estado de cada póliza, en especial, el seguimiento de cobros de las primas de seguros, el reclamo de la planilla a las empresas aseguradoras; as\'i como la comunicación y seguimiento con aseguradoras y asegurados.

\addcontentsline{toc}{subsection}{Proyectos Terminales}
\subsection*{Proyectos Terminales}

\addcontentsline{toc}{subsubsection}{Gestión de Información para el Manejo de Clientes de PyMEs CRM}
\subsubsection*{Gestión de Información para el Manejo de Clientes de PyMEs CRM \cite{pro:crm}}

Sistema  de  seguimiento  que  se  
ejecuta sobre ambiente Web, con el fin de facilitar la intercomunicación con los múltiples clientes 
que puede atender una empresa mediante un foro.

\addcontentsline{toc}{subsubsection}{Proyecto E-File}
\subsubsection*{Proyecto E-File \cite{pro:zurich}}

Sistema  que  permita  a  los  suscriptores  de  una 
aseguradora  crear expedientes de los nuevos clientes, en 
estos   expedientes   se   guardan   los   documentos   empleados   para   realizar  la  cotización  del  negocio  y  que  sirven  como  base  para  
futuras renovaciones, así como para las auditorías que se realizan en la empresa. 

\addcontentsline{toc}{subsubsection}{Proponer mejoras en el área de emisión de pólizas de una aseguradora}
\subsubsection*{Proponer mejoras en el área de emisión de pólizas de una aseguradora \cite{pro:emi}}

En  el  proyecto  de  integración  se  propone  elaborar un manual de procedimientos operativos y un plan de capacitación para agilizar las emisiones de pólizas.

La Tabla \ref{table:related} muestra las principales similitudes y diferencias que los antecedentes tienen con respecto a la propuesta.

\begin{longtabu} to \textwidth {
	X[1,c] X[3,c] X[3,c] }
	\toprule
	\textbf{{Referencia}} & \textbf{{Similitudes}} & \textbf{{Diferencias}} \\
	\hline
	\endfirsthead
	
	\hline
	\textbf{{Referencia}} & \textbf{{Similitudes}} & \textbf{{Diferencias}} \\
	\hline
	\endhead
	\hline
	\endfoot
	\hline
	\caption{Comparación cualitativa de los trabajos relacionados con el proyecto propuesto.}
	\label{table:related}
	\endlastfoot
	
	\cite{www:sicas} &
	\begin{itemize}[leftmargin=*]
		\item Administración de clientes, pólizas, renovaciones, cobranza.
	\end{itemize} &
	\begin{itemize}[leftmargin=*]
		\item Permite la administración de comisiones.
		\item Enfocado a promotores.
		\item No es libre.
		\item Uso complicado.
	\end{itemize} \\

	\midrule
	\cite{www:insly} &
	\begin{itemize}[leftmargin=*]
		\item Administración de clientes, pólizas, pagos.
	\end{itemize} &
	\begin{itemize}[leftmargin=*]
		\item Reportes estadísticos de clientes, ventas, etc.
		\item Enfocado a agencias de seguros.
		\item No es libre.
	\end{itemize} \\
	
	%fila
	\midrule
	\cite{www:asesorestic} &
	\begin{itemize}[leftmargin=*]
		\item Base de datos de clientes, pólizas, cuentas por cobrar, renovaciones.
	\end{itemize} &
	\begin{itemize}[leftmargin=*]
		\item Seguimiento de prospectos.
		\item Reportes.
		\item Seguimiento de cotizaciones, solicitudes y reclamos.
		\item No es libre.
	\end{itemize} \\
	
	%fila
	\midrule
	\cite{pro:crm} &
	\begin{itemize}[leftmargin=*]
		\item Sistema de seguimiento de clientes.
	\end{itemize} &
	\begin{itemize}[leftmargin=*]
		\item Enfocado a la recolección de datos para la toma de decisiones de rentabilidad.
	\end{itemize} \\
	
	%fila
	\midrule
	\cite{pro:zurich} &
	\begin{itemize}[leftmargin=*]
		\item Creación de expedientes de nuevos clientes.
	\end{itemize} &
	\begin{itemize}[leftmargin=*]
		\item Analizar información para analizar riesgos de negocio.
		\item Sistema de uso privado de la aseguradora.
	\end{itemize} \\
	
	
	%fila
	\midrule
	\cite{pro:emi} &
	\begin{itemize}[leftmargin=*]
		\item Mismo dominio de problema.  
	\end{itemize} &
	\begin{itemize}[leftmargin=*]
		\item Trata de mejorar procesos internos de la compañía de seguros.
	\end{itemize} \\	
\end{longtabu}

\section{Descripción Técnica}

El sistema de Control de Cartera se divide en seis m\'odulos.

\addcontentsline{toc}{subsection}{Gestionar clientes}
\subsection*{Gestionar clientes}

Este m\'odulo del sistema ser\'a el encargado de:
\begin{itemize}
	\item Registrar nuevos clientes.
	\item Editar información de los clientes.
	\item Mostrar las pólizas de los clientes.
	\item Mostrar los documentos de los clientes.
	\item Mostrar notificaciones de cumpleaños.
\end{itemize}

\addcontentsline{toc}{subsection}{Gestionar pólizas}
\subsection*{Gestionar pólizas}

Este m\'odulo del sistema ser\'a el encargado de:
\begin{itemize}
	\item Registrar nuevas pólizas
	\item Editar información de las pólizas.
	\item Mostrar pólizas registradas.
	\item Borrar pólizas.
\end{itemize}

\addcontentsline{toc}{subsection}{Gestionar cobranza}
\subsection*{Gestionar cobranza}

Este m\'odulo del sistema ser\'a el encargado de:
\begin{itemize}
	\item Mostrar cobranza pendiente.
	\item Mostrar cobranza cobrada.
	\item Mostrar cobranza vencida.
	\item Actualizar cobranza.
	\item Enviar notificaciones de pago a los clientes.
\end{itemize}

\addcontentsline{toc}{subsection}{Gestionar renovaciones}
\subsection*{Gestionar renovaciones}

Este m\'odulo del sistema ser\'a el encargado de:
\begin{itemize}
	\item Mostrar renovaciones pendientes.
	\item Mostrar renovaciones vencidas.
	\item Actualizar renovaciones.
	\item Enviar notificaciones de renovación a los clientes.
\end{itemize}

\addcontentsline{toc}{subsection}{Gestionar formatos}
\subsection*{Gestionar formatos}

Este m\'odulo del sistema ser\'a el encargado de:
\begin{itemize}
	\item Registrar nuevos formatos.
	\item Enviar formatos.
\end{itemize}

\addcontentsline{toc}{subsection}{Gestionar comisiones}
\subsection*{Reportes de comisiones}

Este m\'odulo del sistema ser\'a el encargado de:
\begin{itemize}
	\item Mostrar proyecciones de comisiones dentro de un periodo de tiempo.
\end{itemize}

\section{Especificación técnica}
El proyecto ser\'a elaborado en el lenguaje Java para que la aplicación sea independiente del sistema operativo. Para el manejo de la base de datos se utilizará Apache Derby \footnote{Base de datos relacional que puede ser embebida en aplicaciones Java \cite{www:derby}} lo que nos permite una aplicación auto contenida y guardar archivos como tipo de datos en las tablas.
Características importantes de la aplicación:
\begin{itemize}
	\item Aplicación de escritorio.
	\item Soporte para un solo usuario.
	\item Respaldo de la base de datos.
\end{itemize}

El proyecto se dará como concluido cuando se pueda concluir una sesión de trabajo utilizando todas las funcionalidades mencionadas en los objetivos específicos y los cambios realizados sean guardados y reflejados en la siguiente sesión.

Al concluir el proyecto de integración se entregará un disco compacto al Coordinador
de Estudios de Ingeniería en Computación que incluirá el reporte final del proyecto
en un archivo PDF (sin restricciones) \footnote{Debe poder visualizarse sin solicitar contraseña}, el código fuente de la aplicación en un archivo comprimido (sin restricciones) \footnote{Debe poder descomprimirse sin solicitar contraseña}. La sección de apéndices del reporte final contendrá al menos un listado del código fuente desarrollado.

Adicionalmente se entregar\'an los artefactos elaborados (casos de uso de texto, diagramas, etc) y manual de usuario.

\section{Cronograma de actividades}

La UEA correspondiente a las actividades que se realizar\'an ser\'a
\begin{itemize}
	\item 1100113 - Proyecto de Integración en Ingeniería en Computación I - 18 créditos.
\end{itemize} 
 Este proyecto se completar\'a en un total de 198 horas. Las actividades se realizar\'an durante el lapso del trimestre académico 2018 - Otoño y se muestran en la Tabla \ref{table:calendar}. Los módulos se realizarán en una sucesión de iteraciones incrementales y evolutivas. Una iteración tiene una duración de 22 horas y est\'a compuesta por las siguientes actividades: 
\begin{enumerate}
	\item Diseño (2 horas)
	\subitem Casos de uso de texto.
	\subitem Diagramas.
	\item Programación (16 horas)
	\subitem Código de producción.
	\subitem Testing.
	\item Demo (2 horas)
	\subitem Retroalimentación.
	\item Refinamiento (2 horas)
	\subitem Recolección de correcciones para la siguiente iteración.
\end{enumerate}

%=================
% para alinear vertical y horizontalmente columnas de iteracion y horas 
% ==============NO FUNCIONA
\newcounter{iteracion}
\newcommand\rownumber{\stepcounter{iteracion}\arabic{iteracion}}

%\newcolumntype{C}{>{\centering\arraybackslash}m{0.1\textwidth} }  %# New column type

%=================

\begin{longtabu} to \textwidth{
		X[3,c] X[1,c] X[3,c]}
	\toprule
	\textbf{Actividad} & \textbf{Horas} & \textbf{Producto} \\
	\hline
	\endfirsthead
	\hline
	\textbf{Actividad} & \textbf{Horas} & \textbf{Producto} \\
	\hline
	\endhead
	\hline
	\endfoot
	\hline
	\textbf{Total:} & \textbf{198} & \\
	\hline
	\caption{Calendario de actividades para el trimestre 2018 Otoño}
	\label{table:calendar}
	\endlastfoot
	
	\begin{itemize}
		\item Análisis de requerimientos.
		\item Identificar casos de uso.
		\item Requerimientos funcionales y no funcionales.
		\item Glosario.
	\end{itemize} & 22 &
	\begin{itemize}
		\item Diagrama de casos de uso.
		\item Glosario.
	\end{itemize} \\ \midrule

	\begin{itemize}
		\item Diagrama del dominio.
		\item Diseño de la base de datos.
		\item Diseño del modelo.
		\item Diseño de la arquitectura.
		\item Creación de casos de uso de texto casuales para los requerimientos principales.
	\end{itemize} & 22 &
	\begin{itemize}
		\item Diagrama del dominio.
		\item Esquema de la base de datos.
		\item Diagrama del modelo.
		\item Diagrama de paquetes.
		\item Casos de uso de texto.
	\end{itemize} \\ \midrule

	\begin{itemize}
		\item  Elaboraci\'on del modulo Gestionar Clientes.
	\end{itemize} & 22 &
	\begin{itemize}
		\item Diagrama de clases.
		\item Código de producci\'on.
	\end{itemize} \\ \midrule

	\begin{itemize}
		\item  Elaboraci\'on del modulo Gestionar Pólizas.
	\end{itemize} & 22 &
	\begin{itemize}
		\item Diagrama de clases.
		\item Código de producci\'on.
	\end{itemize} \\ \midrule

	\begin{itemize}
		\item  Elaboraci\'on del modulo Gestionar Formatos.
	\end{itemize} & 22 &
	\begin{itemize}
		\item Diagrama de clases.
		\item Código de producci\'on.
	\end{itemize} \\ \midrule

	\begin{itemize}
		\item  Elaboraci\'on del modulo Gestionar Cobranza.
	\end{itemize} & 22 &
	\begin{itemize}
		\item Diagrama de clases.
		\item Código de producci\'on.
	\end{itemize} \\ \midrule
	
	\begin{itemize}
		\item  Elaboraci\'on del modulo Gestionar Renovaciones.
	\end{itemize} & 22 &
	\begin{itemize}
		\item Diagrama de clases.
		\item Código de producci\'on.
	\end{itemize} \\ \midrule

	\begin{itemize}
		\item  Elaboraci\'on del modulo Reporte de comisiones.
	\end{itemize} & 22 &
	\begin{itemize}
		\item Diagrama de clases.
		\item Código de producci\'on.
	\end{itemize} \\ \midrule

	\begin{itemize}
		\item Elaboración de manual de usuario.
		\item Elaboraci\'on del reporte final.
		\item Empaquetamiento de la aplicación.
	\end{itemize} & 22 &
	\begin{itemize}
		\item Aplicación finalizada.
		\item Reporte final.
		\item Manual de usuario.	
	\end{itemize}
\end{longtabu}

%
%==============
%\begin{table}[h!]
%	\begin{longtabu} to \textwidth{
%			X[1,c] X[4,c] X[1,c] X[4,c] }
%		\caption{Calendario de actividades para el trimestre 2018 Primavera y 2018 Otoño}
%		\label{table:calendar}
%		\toprule
%		\textbf{Iteración} & \textbf{Actividad} & \textbf{Horas} & \textbf{Producto} \\
%		\hline
%		\endfirsthead
%		\hline
%		\multicolumn{2}{c}{\textbf{Iteración}} & \multicolumn{2}{c}{\textbf{Actividad}} \\
%		\hline
%		\endhead
%		\hline
%		\endfoot
%		\hline
%		\multicolumn{4}{c}{\textbf{Total: 396 hrs}} \\
%		\hline
%		\caption{Calendario de actividades}
%		\label{table:calendar2}
%		\endlastfoot
%		
%		\rownumber & 
%	\begin{itemize}
%		\item Análisis de requerimientos.
%		\item Identificar casos de uso.
%		\item Requerimientos funcionales y no funcionales.
%		\item Glosario.
%	\end{itemize} & 22 &
%\begin{itemize}
%	\item Diagrama de casos de uso.
%	\item Glosario.
%\end{itemize} \\
%
%\midrule
%
%\rownumber & 
%\begin{itemize}
%	\item Diagrama del dominio.
%	\item Diseño de la base de datos.
%	\item Diseño del modelo.
%	\item Diseño de la arquitectura.
%	\item Creación de casos de uso de texto casuales para los requerimientos principales.
%\end{itemize} & 22 &
%\begin{itemize}
%	\item Diagrama del dominio.
%	\item Esquema de la base de datos.
%	\item Diagrama del modelo.
%	\item Diagrama de paquetes.
%	\item Casos de uso de texto.
%\end{itemize} \\
%
%\midrule
%\multicolumn{2}{c}{\textbf{Iteración}} & \multicolumn{2}{c}{\textbf{Actividad}} \\
%
%\midrule
%\multicolumn{4}{c}{\textbf{Gestionar Cliente}} \\
%
%\midrule
%\multicolumn{2}{c}{\multirow{2}{1em}{\rownumber}} & \multicolumn{2}{c}{Registrar}  \\
%& & \multicolumn{2}{c}{Editar} \\
%
%\midrule
%\multicolumn{4}{c}{\textbf{Gestionar Pólizas}} \\
%
%\midrule
%\multicolumn{2}{c}{\multirow{2}{1em}{\rownumber}} & \multicolumn{2}{c}{Registrar}  \\
%& & \multicolumn{2}{c}{Mostrar} \\
%
%\midrule
%\multicolumn{2}{c}{\multirow{2}{1em}{\rownumber}} & \multicolumn{2}{c}{Editar}  \\
%& & \multicolumn{2}{c}{Borrar} \\
%
%\midrule
%\multicolumn{4}{c}{\textbf{Gestionar Formatos}} \\
%
%\midrule
%\multicolumn{2}{c}{\rownumber} & \multicolumn{2}{c}{Registrar} \\
%
%\midrule
%\multicolumn{4}{c}{\textbf{Gestionar Cliente}} \\
%
%\midrule
%\multicolumn{2}{c}{\rownumber} & \multicolumn{2}{c}{Mostrar póliza} \\
%
%\midrule
%\multicolumn{2}{c}{\rownumber} & \multicolumn{2}{c}{Mostrar documentos} \\
%
%\midrule
%\multicolumn{4}{c}{\textbf{Gestionar Cobranza}} \\
%
%\midrule
%\multicolumn{2}{c}{\rownumber} & \multicolumn{2}{c}{Mostrar pendiente} \\
%
%\midrule
%\multicolumn{2}{c}{\multirow{3}{1em}{\rownumber}} & \multicolumn{2}{c}{Actualizar}  \\
%& & \multicolumn{2}{c}{Mostrar cobrada} \\
%& & \multicolumn{2}{c}{Mostrar vencida} \\
%
%\midrule
%\multicolumn{4}{c}{\textbf{Gestionar Renovación}} \\
%
%\midrule
%\multicolumn{2}{c}{\rownumber} & \multicolumn{2}{c}{Mostrar pendientes} \\
%	
%\midrule
%\multicolumn{2}{c}{\multirow{2}{1em}{\rownumber}} & \multicolumn{2}{c}{Actualizar}  \\
%& & \multicolumn{2}{c}{Mostrar vencidas} \\
%
%
%\midrule
%\multicolumn{4}{c}{\textbf{Gestionar Cliente}} \\
%
%\midrule
%\multicolumn{2}{c}{\rownumber} & \multicolumn{2}{c}{Notificación de cumpleaños} \\
%
%\midrule
%\multicolumn{4}{c}{\textbf{Gestionar Renovación}} \\
%
%\midrule
%\multicolumn{2}{c}{\rownumber} & \multicolumn{2}{c}{Enviar notificación} \\
%
%\midrule
%\multicolumn{4}{c}{\textbf{Gestionar Formatos}} \\
%
%\midrule
%\multicolumn{2}{c}{\rownumber} & \multicolumn{2}{c}{Enviar} \\
%
%\midrule
%\multicolumn{4}{c}{\textbf{Comisiones}} \\
%
%\midrule
%\multicolumn{2}{c}{\rownumber} & \multicolumn{2}{c}{Mostrar proyecciones} \\
%
%\midrule
%\multicolumn{4}{c}{\textbf{Finalización}} \\
%
%\midrule
%\multicolumn{2}{c}{\rownumber} & \multicolumn{2}{c}{Testear sistema} \\
%
%\midrule
%\multicolumn{2}{c}{\rownumber} & \multicolumn{2}{c}{Despliegue} \\
%\end{longtabu}


\section{Factibilidad}

\subsection*{Operativa}
Para este proyecto se proponen 22 horas por cada iteración que equivalen aproximadamente a 4 horas diarias trabajando 5 días a la semana. Considero que este tiempo es justo para producir un software robusto y de alta calidad.

\subsection*{Técnica}
No hay restricciones técnicas para el proyecto ya que todo el software y bibliotecas necesarias son gratuitos. Se cuentan con los conocimientos de Java y base de datos necesarios.
Se cuenta con la ayuda de un agente de seguros para la retroalimentación al terminar cada iteración.

\subsection*{Económica}
Estimación de costos mensuales:

\begin{itemize}
	\item Internet = \$830.
	\item Luz = \$150.
	\item Alimento = \$2,500
	\item Sueldo medio de un programador Jr. = \$10,700
\end{itemize}

El costo total por los dos trimestres de duración del proyecto es de \$84,780 pesos.

%\newpage
%\subsection*{Párrafo obligatorio}
El asesor se responsabiliza de guiar al alumno y de que todos los recursos mencionados en la factibilidad técnica estarán disponibles para el alumno, de modo que el proyecto
de integración se pueda concluir en tiempo y forma.\\[2cm]

\begin{center}
	\begin{minipage}{0.4\textwidth}
		\centering
		\begin{tabular}{l}
			\makebox[5cm]{\hrulefill}
		\end{tabular}\\
		Dra. Beatriz Adriana González Beltrán%Grado y nombre del asesor
	\end{minipage}
	\begin{minipage}{0.4\textwidth}
		\centering
		\begin{tabular}{l}
			\makebox[5cm]{\hrulefill}
		\end{tabular}\\
		Dra. Sonia G. Mendoza Chapa%Grado y nombre del asesor
	\end{minipage}
\end{center}

