%Este comando hace que la página que contiene la introducción sea la número uno
\setcounter{page}{1}

\section*{Introducción}%Introducción
Un agente de seguros es la persona física o moral autorizada por la Comisión Nacional de Seguros y Fianzas para realizar actividades de intermediación en la contratación de seguros o de fianzas. Las actividades de intermediación que pueden realizar los agentes consiste en el intercambio de propuestas, comercialización y asesoramiento para la contratación de seguros o fianzas, su conservación o modificación, renovación o cancelación. \cite{www:reg-agentes}

A medida del aumento de clientes que asesora el agente el manejo de las pólizas se dificulta.
El Control de Cartera para Agentes de Seguros facilita esta tarea.

\section*{Justificación}

Los agentes de seguros no cuentan con un software libre para este aspecto del trabajo, por lo tanto, tienen que recurrir a sus propios métodos. La mayoría utiliza hojas de calculo, otros un sistema de archivos y el resto no tienen 
un proceso establecido.
El Control de Cartera es un software libre que facilita la gestión de clientes, pólizas, cobranzas y renovaciones; lo cual tiene un impacto en la productividad del agente.

\section*{Objetivos}

\subsection*{Objetivo General}
Facilitar y el control de cartera para agentes de seguros.

\subsection*{Objetivos Específicos}
\begin{enumerate}
  \item Gestionar clientes.
  \subitem Agregar, modificar, borrar.
  \item Gestionar pólizas
    \subitem Agregar, modificar, borrar.
  \item Gestionar cobranza.
  \subitem Ver cobranza pendiente.
  \subitem Actualizar cobranza.
  \item Gestionar renovaciones.
  \subitem Ver renovaciones pendientes.
  \subitem Actualizar renovaciones.
\end{enumerate}

\section*{Trabajos Relacionados}
\subsection*{Software Comercial}
\subsubsection*{SICAS \cite{www:sicas}} 

SICAS Online es un sistema WEB para el Control y Administración para Cartera de Agentes, Corredores, Promotores de Seguros y afines. Contempla una lógica de negocio para solventar las necesidades mas básicas o complejas que se puedan presentar. Este software esta orientado mas hacia los promotores.

\subsubsection*{Insly \cite{www:insly}}

Software para agencias de seguros basado en la nube. Permite administrar el flujo de ventas gestionando clientes, pólizas y productos de seguros.

\subsubsection*{Asesorestic – Software para Administración de pólizas de seguros \cite{www:asesorestic}}

El sistema contribuye a tener el control sobre el estado de cada póliza, en especial, el seguimiento de cobros de las primas de seguros, el reclamo de la planilla a las empresas aseguradoras y la comunicación y seguimiento con aseguradoras y asegurados.

\begin{table}[H] %ht!
  \begin{tabular}{p{0.15\textwidth} p{0.4\textwidth} p{0.4\textwidth}}
    \toprule
    \textbf{{Referencia}} & \textbf{{Similitudes}} & \textbf{{Diferencias}} \\
    \toprule
    SICAS Online &
    \begin{itemize}[leftmargin=*]
        \item Administración de clientes, pólizas, renovaciones, cobranza.
    \end{itemize} &
    \begin{itemize}[leftmargin=*]
        \item Administración de comisiones.
    \end{itemize} \\
    \midrule
    Insly &
    \begin{itemize}[leftmargin=*]
        \item Administración de clientes, pólizas, pagos.
    \end{itemize} &
    \begin{itemize}[leftmargin=*]
        \item Reportes estadísticos de clientes, ventas, etc.
    \end{itemize} \\
\midrule
Asesorestic – Software para Administración de pólizas de seguros &
\begin{itemize}[leftmargin=*]
	\item Base de datos de clientes, pólizas, cuentas por cobrar, renovaciones.
\end{itemize} &
\begin{itemize}[leftmargin=*]
	\item Seguimiento de prospectos.
	\item Reportes.
	\item Seguimiento de cotizaciones, solicitudes y reclamos.
\end{itemize} \\

    \bottomrule
  \end{tabular}
  \caption{Comparación cualitativa de los trabajos relacionados con el proyecto propuesto.}
  \label{table:related}
\end{table}
