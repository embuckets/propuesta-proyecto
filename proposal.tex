%Este comando hace que la página que contiene la introducción sea la número uno
\setcounter{page}{1}
\tableofcontents

\section{Introducción}%Introducción
Un agente de seguros es la persona física o moral autorizada por la Comisión Nacional de Seguros y Fianzas para realizar actividades de intermediación en la contratación de seguros o de fianzas. Las actividades de intermediación que pueden realizar los agentes consiste en el intercambio de propuestas, comercialización y asesoramiento para la contratación de seguros o fianzas, su conservación o modificación, renovación o cancelación. \cite{www:reg-agentes}

A medida del aumento de clientes que asesora el agente el manejo de las pólizas se dificulta.
El Control de Cartera para Agentes de Seguros facilita esta tarea.

\section{Justificación}

Los agentes de seguros no cuentan con un software libre para este aspecto del trabajo, por lo tanto, tienen que recurrir a sus propios métodos. La mayoría utiliza hojas de calculo, otros un sistema de archivos y el resto no tienen 
un proceso establecido.
El Control de Cartera es un software libre que facilita la gestión de clientes, pólizas, cobranzas y renovaciones; lo cual tiene un impacto en la productividad del agente.

\section{Objetivos}
\addcontentsline{toc}{subsection}{Objetivo General}
\subsection*{Objetivo General}
Facilitar y el control de cartera para agentes de seguros.
\addcontentsline{toc}{subsection}{Objetivos Específicos}
\subsection*{Objetivos Específicos}
\begin{enumerate}
  \item Gestionar clientes.
  \item Gestionar pólizas
  \item Gestionar cobranza.
  \item Gestionar renovaciones.
  \item Gestionar formatos.
  \item Reportes de comisiones.
\end{enumerate}

\section{Trabajos Relacionados}
\addcontentsline{toc}{subsection}{Software Comercial}
\subsection*{Software Comercial}
\addcontentsline{toc}{subsubsection}{SICAS}
\subsubsection*{SICAS \cite{www:sicas}} 

SICAS Online es un sistema WEB para el Control y Administración para Cartera de Agentes, Corredores, Promotores de Seguros y afines. Contempla una lógica de negocio para solventar las necesidades mas básicas o complejas que se puedan presentar. Este software esta orientado mas hacia los promotores.

\addcontentsline{toc}{subsubsection}{Insly}
\subsubsection*{Insly \cite{www:insly}}

Software para agencias de seguros basado en la nube. Permite administrar el flujo de ventas gestionando clientes, pólizas y productos de seguros.

\addcontentsline{toc}{subsubsection}{Asesorestic}
\subsubsection*{Asesorestic – Software para Administración de pólizas de seguros \cite{www:asesorestic}}

El sistema contribuye a tener el control sobre el estado de cada póliza, en especial, el seguimiento de cobros de las primas de seguros, el reclamo de la planilla a las empresas aseguradoras y la comunicación y seguimiento con aseguradoras y asegurados.

\addcontentsline{toc}{subsection}{Proyectos Terminales}
\subsection*{Proyectos Terminales}

\addcontentsline{toc}{subsubsection}{Gestión de Información para el Manejo de Clientes de PyMEs CRM}
\subsubsection*{Gestión de Información para el Manejo de Clientes de PyMEs CRM \cite{pro:crm}}

Sistema  de  seguimiento  que  se  
ejecuta sobre ambiente Web, con el fin de facilitar la intercomunicación con los múltiples clientes 
que puede atender una empresa.

\addcontentsline{toc}{subsubsection}{Proyecto E-File}
\subsubsection*{Proyecto E-File \cite{pro:zurich}}

Sistema  que  permita  a  los  suscriptores  de  la 
aseguradora  crear expedientes de los nuevos clientes   , en 
estas   carpetas   se   guardaran   los   documentos   empleados   para   
realizar  la  cotización  del  negocio  y  que  sirven  como  base  para  
futuras 
renovaciones, 
así como para las auditorías que se realizan 
en la empresa. 

\addcontentsline{toc}{subsubsection}{Proponer mejoras en el área de emisión de pólizas de una aseguradora}
\subsubsection*{Proponer mejoras en el área de emisión de pólizas de una aseguradora \cite{pro:emi}}

En  el  proyecto  de  integración  se  propone  elaborar un manual de procedimientos operativos y un plan de capacitación para agilizar las emisiones de pólizas.

\begin{table}[ht!] %ht!
  \begin{tabular}{p{0.15\textwidth} p{0.4\textwidth} p{0.4\textwidth}}
    \toprule
    \textbf{{Referencia}} & \textbf{{Similitudes}} & \textbf{{Diferencias}} \\
    \toprule
    %fila
    SICAS Online &
    \begin{itemize}[leftmargin=*]
        \item Administración de clientes, pólizas, renovaciones, cobranza.
    \end{itemize} &
    \begin{itemize}[leftmargin=*]
        \item Administración de comisiones.
    \end{itemize} \\
    \midrule
    
    %fila
    Insly &
    \begin{itemize}[leftmargin=*]
        \item Administración de clientes, pólizas, pagos.
    \end{itemize} &
    \begin{itemize}[leftmargin=*]
        \item Reportes estadísticos de clientes, ventas, etc.
    \end{itemize} \\

%fila
\midrule
Asesorestic – Software para Administración de pólizas de seguros &
\begin{itemize}[leftmargin=*]
	\item Base de datos de clientes, pólizas, cuentas por cobrar, renovaciones.
\end{itemize} &
\begin{itemize}[leftmargin=*]
	\item Seguimiento de prospectos.
	\item Reportes.
	\item Seguimiento de cotizaciones, solicitudes y reclamos.
\end{itemize} \\

%fila
\midrule
Gestión de Información para el Manejo de Clientes de PyMEs CRM &
\begin{itemize}[leftmargin=*]
	\item Sistema de seguimiento de clientes.
\end{itemize} &
\begin{itemize}[leftmargin=*]
	\item Enfocado a la recolección de datos para la toma de decisiones de rentabilidad.
\end{itemize} \\

%fila
\midrule
Proyecto E-File &
\begin{itemize}[leftmargin=*]
	\item Creación de expedientes de nuevos clientes.
\end{itemize} &
\begin{itemize}[leftmargin=*]
	\item Analizar información para analizar riesgos de negocio.
\end{itemize} \\


%fila
\midrule
Proponer mejoras en el área de emisión de pólizas de una aseguradora &
\begin{itemize}[leftmargin=*]
	\item Mismo dominio de problema.  
\end{itemize} &
\begin{itemize}[leftmargin=*]
	\item Trata de mejorar procesos internos de la compañía de seguros.
\end{itemize} \\


    \bottomrule
  \end{tabular}
  \caption{Comparación cualitativa de los trabajos relacionados con el proyecto propuesto.}
  \label{table:related}
\end{table}

\section{Descripción Técnica}

\addcontentsline{toc}{subsection}{Gestionar Clientes}
\subsection*{Gestionar Clientes}

Este modulo del sistema sera el encargado de:
\begin{itemize}
	\item Registrar nuevos clientes.
	\item Editar información de los clientes.
	\item Mostrar las pólizas de los clientes.
	\item Mostrar los documentos de los clientes.
	\item Mostrar notificaciones de cumpleaños.
\end{itemize}

\addcontentsline{toc}{subsection}{Gestionar pólizas}
\subsection*{Gestionar pólizas}

Este modulo del sistema sera el encargado de:
\begin{itemize}
	\item Registrar nuevas pólizas
	\item Editar información de las pólizas.
	\item Mostrar pólizas registradas.
	\item Borrar pólizas.
\end{itemize}

\addcontentsline{toc}{subsection}{Gestionar cobranza}
\subsection*{Gestionar cobranza}

Este modulo del sistema sera el encargado de:
\begin{itemize}
	\item Mostrar cobranza pendiente.
	\item Mostrar cobranza cobrada.
	\item Mostrar cobranza vencida.
	\item Actualizar cobranza.
	\item Enviar notificaciones de pago a los clientes.
\end{itemize}

\addcontentsline{toc}{subsection}{Gestionar renovaciones}
\subsection*{Gestionar renovaciones}

Este modulo del sistema sera el encargado de:
\begin{itemize}
	\item Mostrar renovaciones pendientes.
	\item Mostrar renovaciones vencidas.
	\item Actualizar renovaciones.
	\item Enviar notificaciones de renovación a los clientes.
\end{itemize}

\addcontentsline{toc}{subsection}{Gestionar formatos}
\subsection*{Gestionar formatos}

Este modulo del sistema sera el encargado de:
\begin{itemize}
	\item Registrar nuevos formatos.
	\item Enviar formatos.
\end{itemize}

\addcontentsline{toc}{subsection}{Gestionar comisiones}
\subsection*{Reportes de comisiones}

Este modulo del sistema sera el encargado de:
\begin{itemize}
	\item Mostrar proyecciones de comisiones dentro de un periodo de tiempo.
\end{itemize}

\section{Especificación técnica}
El proyecto sera elaborado en el lenguaje Java para que la aplicación sea independiente del sistema operativo. Para el manejo de la base de datos se utilizará Apache Derby \footnote{Base de datos relacional que puede ser embebida en aplicaciones Java \cite{www:derby}} lo que nos permite una aplicación auto contenida y guardar archivos como tipo de datos en las tablas.
Características importantes de la aplicación:
\begin{itemize}
	\item Aplicación de escritorio.
	\item Soporte para un solo usuario.
	\item Respaldo de la base de datos.
\end{itemize}

El proyecto se dará como concluido cuando se pueda concluir una sesión de trabajo utilizando todas las funcionalidades mencionadas en los objetivos específicos y los cambios realizados sean guardados y reflejados en la siguiente sesión.

Al concluir el proyecto de integración se entregará un disco compacto al Coordinador
de Estudios de Ingeniería en Computación que incluirá el reporte final del proyecto
en un archivo PDF (sin restricciones) \footnote{Debe poder visualizarse sin solicitar contraseña}, el código fuente de la aplicación en un archivo comprimido (sin restricciones) \footnote{Debe poder descomprimirse sin solicitar contraseña}. La sección de apéndices del reporte final contendrá al
menos un listado del código fuente desarrollado.

