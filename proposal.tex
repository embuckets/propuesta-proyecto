%Este comando hace que la página que contiene la introducción sea la número uno
\setcounter{page}{1}
\tableofcontents

\section{Introducción}%Introducción
Un agente de seguros es la persona física o moral autorizada por la Comisión Nacional de Seguros y Fianzas para realizar actividades de intermediación en la contratación de seguros o de fianzas. Las actividades de intermediación que pueden realizar los agentes consiste en el intercambio de propuestas, comercialización y asesoramiento para la contratación de seguros o fianzas, su conservación o modificación, renovación o cancelación. \cite{www:reg-agentes}

Existe una gran variedad de ramos de pólizas (vida, gastos médicos, autos, daños, etc) cada uno con condiciones y coberturas distintas. El asegurado puede elegir el tipo de pago fraccionado (mensual, trimestres, semestral, anual) y en algunas pólizas la vigencia del plan.

El agente debe asesorar al asegurado en el uso de su póliza, ademas de recordarle el pago de su recibo y fecha de renovación.

Con los años los agentes de seguros pueden llegar a tener cientos de clientes y llevar el control de las pólizas se vuelve complicado.

El Control de Cartera para Agentes de Seguros facilita esta tarea.

\section{Justificación}

Los agentes de seguros no cuentan con un software libre para este aspecto del trabajo, por lo tanto, tienen que recurrir a sus propios métodos. La mayoría utiliza hojas de calculo o un sistema de archivos, lo que hace el mantenimiento de cartera y los datos una tarea muy manual que quita tiempo.
El Control de Cartera es un software libre que facilita la gestión de clientes, pólizas, cobranzas y renovaciones; lo cual tiene un impacto en la productividad del agente.

\section{Objetivos}
\addcontentsline{toc}{subsection}{Objetivo General}
\subsection*{Objetivo General}
Desarrollar un Sistema de Control de Cartera para agentes de seguros que disminuirá la complejidad de la gestión de cartera y aumentará la productividad del agente.
\addcontentsline{toc}{subsection}{Objetivos Específicos}
\subsection*{Objetivos Específicos}
\begin{enumerate}
	\item Gestionar clientes, pólizas, cobranza, renovaciones.
	\item Guardar documentos de los clientes y formatos de las aseguradoras.
	\item Reportar proyecciones de comisiones.
\end{enumerate}

\section{Trabajos Relacionados}
\addcontentsline{toc}{subsection}{Software Comercial}
\subsection*{Software Comercial}
\addcontentsline{toc}{subsubsection}{SICAS}
\subsubsection*{SICAS \cite{www:sicas}} 

SICAS Online es un sistema WEB para el Control y Administración para Cartera de Agentes, Corredores, Promotores de Seguros y afines. Contempla una lógica de negocio para solventar las necesidades mas básicas o complejas que se puedan presentar. Este software esta orientado mas hacia los promotores.

\addcontentsline{toc}{subsubsection}{Insly}
\subsubsection*{Insly \cite{www:insly}}

Software para agencias de seguros basado en la nube. Permite administrar el flujo de ventas gestionando clientes, pólizas y productos de seguros.

\addcontentsline{toc}{subsubsection}{Asesorestic}
\subsubsection*{Asesorestic – Software para Administración de pólizas de seguros \cite{www:asesorestic}}

El sistema contribuye a tener el control sobre el estado de cada póliza, en especial, el seguimiento de cobros de las primas de seguros, el reclamo de la planilla a las empresas aseguradoras y la comunicación y seguimiento con aseguradoras y asegurados.

\addcontentsline{toc}{subsection}{Proyectos Terminales}
\subsection*{Proyectos Terminales}

\addcontentsline{toc}{subsubsection}{Gestión de Información para el Manejo de Clientes de PyMEs CRM}
\subsubsection*{Gestión de Información para el Manejo de Clientes de PyMEs CRM \cite{pro:crm}}

Sistema  de  seguimiento  que  se  
ejecuta sobre ambiente Web, con el fin de facilitar la intercomunicación con los múltiples clientes 
que puede atender una empresa.

\addcontentsline{toc}{subsubsection}{Proyecto E-File}
\subsubsection*{Proyecto E-File \cite{pro:zurich}}

Sistema  que  permita  a  los  suscriptores  de  la 
aseguradora  crear expedientes de los nuevos clientes   , en 
estas   carpetas   se   guardaran   los   documentos   empleados   para   
realizar  la  cotización  del  negocio  y  que  sirven  como  base  para  
futuras 
renovaciones, 
así como para las auditorías que se realizan 
en la empresa. 

\addcontentsline{toc}{subsubsection}{Proponer mejoras en el área de emisión de pólizas de una aseguradora}
\subsubsection*{Proponer mejoras en el área de emisión de pólizas de una aseguradora \cite{pro:emi}}

En  el  proyecto  de  integración  se  propone  elaborar un manual de procedimientos operativos y un plan de capacitación para agilizar las emisiones de pólizas.

La Tabla \ref{table:related} muestra las principales similitudes y diferencias que los antecedentes tienen con respecto a la propuesta.

\begin{table}[h!] %ht!
  \begin{tabular}{p{0.2\textwidth} p{0.4\textwidth} p{0.4\textwidth}}
    \toprule
    \textbf{{Referencia}} & \textbf{{Similitudes}} & \textbf{{Diferencias}} \\
    \toprule
    %fila
    SICAS Online &
    \begin{itemize}[leftmargin=*]
        \item Administración de clientes, pólizas, renovaciones, cobranza.
    \end{itemize} &
    \begin{itemize}[leftmargin=*]
        \item Administración de comisiones.
    \end{itemize} \\
    \midrule
    
    %fila
    Insly &
    \begin{itemize}[leftmargin=*]
        \item Administración de clientes, pólizas, pagos.
    \end{itemize} &
    \begin{itemize}[leftmargin=*]
        \item Reportes estadísticos de clientes, ventas, etc.
    \end{itemize} \\

%fila
\midrule
Asesorestic – Software para Administración de pólizas de seguros &
\begin{itemize}[leftmargin=*]
	\item Base de datos de clientes, pólizas, cuentas por cobrar, renovaciones.
\end{itemize} &
\begin{itemize}[leftmargin=*]
	\item Seguimiento de prospectos.
	\item Reportes.
	\item Seguimiento de cotizaciones, solicitudes y reclamos.
\end{itemize} \\

%fila
\midrule
Gestión de Información para el Manejo de Clientes de PyMEs CRM &
\begin{itemize}[leftmargin=*]
	\item Sistema de seguimiento de clientes.
\end{itemize} &
\begin{itemize}[leftmargin=*]
	\item Enfocado a la recolección de datos para la toma de decisiones de rentabilidad.
\end{itemize} \\

%fila
\midrule
Proyecto E-File &
\begin{itemize}[leftmargin=*]
	\item Creación de expedientes de nuevos clientes.
\end{itemize} &
\begin{itemize}[leftmargin=*]
	\item Analizar información para analizar riesgos de negocio.
\end{itemize} \\


%fila
\midrule
Proponer mejoras en el área de emisión de pólizas de una aseguradora &
\begin{itemize}[leftmargin=*]
	\item Mismo dominio de problema.  
\end{itemize} &
\begin{itemize}[leftmargin=*]
	\item Trata de mejorar procesos internos de la compañía de seguros.
\end{itemize} \\
    \bottomrule
  \end{tabular}
  \caption{Comparación cualitativa de los trabajos relacionados con el proyecto propuesto.}
  \label{table:related}
\end{table}

\section{Descripción Técnica}

\addcontentsline{toc}{subsection}{Gestionar Clientes}
\subsection*{Gestionar Clientes}

Este modulo del sistema sera el encargado de:
\begin{itemize}
	\item Registrar nuevos clientes.
	\item Editar información de los clientes.
	\item Mostrar las pólizas de los clientes.
	\item Mostrar los documentos de los clientes.
	\item Mostrar notificaciones de cumpleaños.
\end{itemize}

\addcontentsline{toc}{subsection}{Gestionar pólizas}
\subsection*{Gestionar pólizas}

Este modulo del sistema sera el encargado de:
\begin{itemize}
	\item Registrar nuevas pólizas
	\item Editar información de las pólizas.
	\item Mostrar pólizas registradas.
	\item Borrar pólizas.
\end{itemize}

\addcontentsline{toc}{subsection}{Gestionar cobranza}
\subsection*{Gestionar cobranza}

Este modulo del sistema sera el encargado de:
\begin{itemize}
	\item Mostrar cobranza pendiente.
	\item Mostrar cobranza cobrada.
	\item Mostrar cobranza vencida.
	\item Actualizar cobranza.
	\item Enviar notificaciones de pago a los clientes.
\end{itemize}

\addcontentsline{toc}{subsection}{Gestionar renovaciones}
\subsection*{Gestionar renovaciones}

Este modulo del sistema sera el encargado de:
\begin{itemize}
	\item Mostrar renovaciones pendientes.
	\item Mostrar renovaciones vencidas.
	\item Actualizar renovaciones.
	\item Enviar notificaciones de renovación a los clientes.
\end{itemize}

\addcontentsline{toc}{subsection}{Gestionar formatos}
\subsection*{Gestionar formatos}

Este modulo del sistema sera el encargado de:
\begin{itemize}
	\item Registrar nuevos formatos.
	\item Enviar formatos.
\end{itemize}

\addcontentsline{toc}{subsection}{Gestionar comisiones}
\subsection*{Reportes de comisiones}

Este modulo del sistema sera el encargado de:
\begin{itemize}
	\item Mostrar proyecciones de comisiones dentro de un periodo de tiempo.
\end{itemize}

\section{Especificación técnica}
El proyecto sera elaborado en el lenguaje Java para que la aplicación sea independiente del sistema operativo. Para el manejo de la base de datos se utilizará Apache Derby \footnote{Base de datos relacional que puede ser embebida en aplicaciones Java \cite{www:derby}} lo que nos permite una aplicación auto contenida y guardar archivos como tipo de datos en las tablas.
Características importantes de la aplicación:
\begin{itemize}
	\item Aplicación de escritorio.
	\item Soporte para un solo usuario.
	\item Respaldo de la base de datos.
\end{itemize}

El proyecto se dará como concluido cuando se pueda concluir una sesión de trabajo utilizando todas las funcionalidades mencionadas en los objetivos específicos y los cambios realizados sean guardados y reflejados en la siguiente sesión.

Al concluir el proyecto de integración se entregará un disco compacto al Coordinador
de Estudios de Ingeniería en Computación que incluirá el reporte final del proyecto
en un archivo PDF (sin restricciones) \footnote{Debe poder visualizarse sin solicitar contraseña}, el código fuente de la aplicación en un archivo comprimido (sin restricciones) \footnote{Debe poder descomprimirse sin solicitar contraseña}. La sección de apéndices del reporte final contendrá al menos un listado del código fuente desarrollado.

Adicionalmente se entregaran los artefactos elaborados (casos de uso de texto, diagramas, etc) y manual de usuario.

\section{Cronograma de actividades}

Las UEA's correspondientes a las actividades que se realizaran serán:
\begin{itemize}
	\item 1100113 - Proyecto de Integración en Ingeniería en Computación I - 18 créditos.
	\item 1100123 - Proyecto de Integración en Ingeniería en Computación II - 18 créditos.
\end{itemize} 
 Este proyecto se completara en un total de 396 horas. Las actividades se realizaran durante el lapso de los trimestres académicos 2018-Primavera y 2018-Otoño. El proyecto se realizará en una sucesión de iteraciones incrementales y evolutivas. Una iteración tiene una duración de 22 horas y esta compuesta por las siguientes actividades: 
\begin{enumerate}
	\item Diseño (2 horas)
	\subitem Casos de uso de texto.
	\subitem Diagramas.
	\item Programación (16 horas)
	\subitem Código de producción.
	\subitem Testing.
	\item Demo (2 horas)
	\subitem Retroalimentación.
	\item Refinamiento (2 horas)
	\subitem Recolección de correcciones para la siguiente iteración.
\end{enumerate}

Se realizarán 2 o 3 iteraciones por cada uno de los apartados en la descripción técnica. Cada iteración se concentrará un un subconjunto de funcionalidades dentro de los objetivos específicos. 
Las actividades para el trimestre 2018 - P se muestran en la Tabla \ref{table:calendarP}

%=================
% para alinear vertical y horizontalmente columnas de iteracion y horas 
% ==============NO FUNCIONA
\newcounter{iteracion}
\newcommand\rownumber{\stepcounter{iteracion}\arabic{iteracion}}

\newcolumntype{C}{>{\centering\arraybackslash}m{0.1\textwidth} }  %# New column type

%=================
%\begin{table}[h!]
	\begin{longtabu} to \textwidth{
		C p{0.4\textwidth} C p{0.4\textwidth} }
		%\caption{Calendario de actividades para el trimestre 2018 Primavera}
		%\label{table:calendarP}
		\toprule
		\textbf{Iteración} & \textbf{Actividad} & \textbf{Horas} & \textbf{Producto} \\
		\hline
		\endfirsthead
		\hline
		\multicolumn{2}{c}{\textbf{Iteración}} & \multicolumn{2}{c}{\textbf{Actividad}} \\
		\hline
		\endhead
		\hline
		\endfoot
		\hline
		\multicolumn{4}{c}{\textbf{Total: 396 hrs}} \\
		\hline
		
		\endlastfoot
		
		\rownumber & 
	\begin{itemize}
		\item Análisis de requerimientos.
		\item Identificar casos de uso.
		\item Requerimientos funcionales y no funcionales.
		\item Glosario.
	\end{itemize} & 22 &
\begin{itemize}
	\item Diagrama de casos de uso.
	\item Glosario.
\end{itemize} \\

\midrule

\rownumber & 
\begin{itemize}
	\item Diagrama del dominio.
	\item Diseño de la base de datos.
	\item Diseño del modelo.
	\item Diseño de la arquitectura.
	\item Creación de casos de uso de texto casuales para los requerimientos principales.
\end{itemize} & 22 &
\begin{itemize}
	\item Diagrama del dominio.
	\item Esquema de la base de datos.
	\item Diagrama del modelo.
	\item Diagrama de paquetes.
	\item Casos de uso de texto.
\end{itemize} \\

\midrule
\multicolumn{2}{c}{\textbf{Iteración}} & \multicolumn{2}{c}{\textbf{Actividad}} \\

\midrule
\multicolumn{4}{c}{\textbf{Gestionar Cliente}} \\

\midrule
\multicolumn{2}{c}{\multirow{2}{1em}{\rownumber}} & \multicolumn{2}{c}{Registrar}  \\
& & \multicolumn{2}{c}{Editar} \\

\midrule
\multicolumn{4}{c}{\textbf{Gestionar Pólizas}} \\

\midrule
\multicolumn{2}{c}{\multirow{2}{1em}{\rownumber}} & \multicolumn{2}{c}{Registrar}  \\
& & \multicolumn{2}{c}{Mostrar} \\

\midrule
\multicolumn{2}{c}{\multirow{2}{1em}{\rownumber}} & \multicolumn{2}{c}{Editar}  \\
& & \multicolumn{2}{c}{Borrar} \\

\midrule
\multicolumn{4}{c}{\textbf{Gestionar Formatos}} \\

\midrule
\multicolumn{2}{c}{\rownumber} & \multicolumn{2}{c}{Registrar} \\

\midrule
\multicolumn{4}{c}{\textbf{Gestionar Cliente}} \\

\midrule
\multicolumn{2}{c}{\rownumber} & \multicolumn{2}{c}{Mostrar póliza} \\

\midrule
\multicolumn{2}{c}{\rownumber} & \multicolumn{2}{c}{Mostrar documentos} \\

\midrule
\multicolumn{4}{c}{\textbf{Gestionar Cobranza}} \\

\midrule
\multicolumn{2}{c}{\rownumber} & \multicolumn{2}{c}{Mostrar pendiente} \\

\midrule
\multicolumn{2}{c}{\multirow{3}{1em}{\rownumber}} & \multicolumn{2}{c}{Actualizar}  \\
& & \multicolumn{2}{c}{Mostrar cobrada} \\
& & \multicolumn{2}{c}{Mostrar vencida} \\

\midrule
\multicolumn{4}{c}{\textbf{Gestionar Renovación}} \\

\midrule
\multicolumn{2}{c}{\rownumber} & \multicolumn{2}{c}{Mostrar pendientes} \\
	
\midrule
\multicolumn{2}{c}{\multirow{2}{1em}{\rownumber}} & \multicolumn{2}{c}{Actualizar}  \\
& & \multicolumn{2}{c}{Mostrar vencidas} \\


\midrule
\multicolumn{4}{c}{\textbf{Gestionar Cliente}} \\

\midrule
\multicolumn{2}{c}{\rownumber} & \multicolumn{2}{c}{Notificación de cumpleaños} \\

\midrule
\multicolumn{4}{c}{\textbf{Gestionar Renovación}} \\

\midrule
\multicolumn{2}{c}{\rownumber} & \multicolumn{2}{c}{Enviar notificación} \\

\midrule
\multicolumn{4}{c}{\textbf{Gestionar Formatos}} \\

\midrule
\multicolumn{2}{c}{\rownumber} & \multicolumn{2}{c}{Enviar} \\

\midrule
\multicolumn{4}{c}{\textbf{Comisiones}} \\

\midrule
\multicolumn{2}{c}{\rownumber} & \multicolumn{2}{c}{Mostrar proyecciones} \\

\midrule
\multicolumn{4}{c}{\textbf{Finalización}} \\

\midrule
\multicolumn{2}{c}{\rownumber} & \multicolumn{2}{c}{Testear sistema} \\

\midrule
\multicolumn{2}{c}{\rownumber} & \multicolumn{2}{c}{Despliegue} \\

\end{longtabu}

\section{Factibilidad}

\subsection*{Operativa}
Para este proyecto propuse 22 horas por iteración que equivalen aproximadamente a 4 horas diarias trabajando 5 días a la semana. Considero que este tiempo es justo para producir un software robusto y de alta calidad.

\subsection*{Técnica}
No hay restricciones técnicas para el proyecto ya que todo el software y librerías necesarios son gratuitos. Se cuentan con los conocimientos de java y base de datos necesarios.
Se cuenta con la ayuda de un agente de seguros para la retroalimentación al terminar cada iteración.

\subsection*{Económica}
Estimación de costos mensuales:

\begin{itemize}
	\item Internet = \$830.
	\item Luz = \$150.
	\item Alimento = \$2,500
	\item Sueldo medio de un programador Jr. = \$10,700
\end{itemize}

El costo total por los dos trimestres de duración del proyecto es de \$84,780 pesos.

%\subsection*{Párrafo obligatorio}
El asesor se responsabiliza de guiar al alumno y de que todos los recursos mencionados en la factibilidad técnica estarán disponibles para el alumno, de modo que el proyecto
de integración se pueda concluir en tiempo y forma.

\begin{center}
	\begin{minipage}{0.4\textwidth}
		\centering
		\begin{tabular}{l}
			\makebox[5cm]{\hrulefill}
		\end{tabular}\\
		Dra. Beatriz Adriana González Beltrán%Grado y nombre del asesor
	\end{minipage}
	\begin{minipage}{0.4\textwidth}
		\centering
		\begin{tabular}{l}
			\makebox[5cm]{\hrulefill}
		\end{tabular}\\
		Dra. Sonia G. Mendoza Chapa%Grado y nombre del asesor
	\end{minipage}
\end{center}

